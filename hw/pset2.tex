\documentclass[12pt,letterpaper,cm]{hmcpset}
\usepackage[margin=1in]{geometry}
\usepackage{graphicx}
\usepackage{amsmath,amssymb}
\usepackage{enumerate,hyperref}

% info for header block in upper right hand corner
\name{\_\_\_\_\_\_}
\class{Math 173}
\assignment{Problem Set 2}
\duedate{September 24, 2018}
\setlength\parindent{0pt}

\begin{document}

\textbf{Note:} The Extra Credit (EC) problem
is worth as much as a regular problem; even if you don't attempt it, you should read it thoroughly
since these problems will hint at future directions in the course or other interesting topics.

\begin{problem}[1 (Lax 3.14)]
    Let $\mathcal{V}$ be a vector space of arbitrary dimension and $T : \mathcal{V}\to\mathcal{V}$
    be rank $1$.
\begin{enumerate}[(a)]
    \item Show there exists a unique number $c$ such that $T^2 = cT$.
    \item Assuming $\dim\mathcal{V}<\infty$, show that $c\neq 1$ implies $I-T$ has an inverse.
\end{enumerate}
\end{problem}

\begin{solution}
    \vfill
\end{solution}

\begin{problem}[2]
    Let $\mathcal{V}$ be a vector space, and recall that two linear spaces are
    isomorphic if there exists an invertible linear map from one to the other.
\begin{enumerate}[(a)]
    \item If $\mathcal{V}$ is infinite dimensional, prove that $\dim\mathcal{V} < \dim\mathcal{V}'$,
        or in other words that there exists a linear injection from $\mathcal{V}$ to $\mathcal{V}'$ that is not a surjection.
    \item On the other hand, if $\mathcal{V}$ is finite dimensional prove that
        $\mathcal{V}$ is isomorphic to $\mathcal{V}'$.
\end{enumerate}
\end{problem}

\begin{solution}
    \vfill
\end{solution}
\clearpage

\begin{problem}[3]
    Let $A\in\mathbb{C}^{m\times n}$ and decompose $A = U \Sigma V^*$ into its singular value
    decomposition where $\Sigma$ is the diagonal matrix composed of the singular values
    $\sigma_1\geq\sigma_2\geq \dots\geq 0$. Write $\Sigma_k = \operatorname{diag}\{\sigma_1,\sigma_2
    ,\ldots,\sigma_k\}$ for $k=1,2,\ldots,\max\{m,n\}$ and define the truncated singular value decomposition
    $A_k = U\Sigma_k V^*$. Prove\footnote{As we'll see later in the course, this says that the distance between
    $A$ and $A_k$ in the so-called \emph{spectral norm} is $\sigma_{k+1}$. It turns out that this is the
    \emph{best} rank-$k$ approximation to $A$ in both the spectral and Frobenius norms.}
    that the largest singular value of $A - A_k$ is $\sigma_{k+1}$.
\end{problem}

\begin{solution}
    \vfill
\end{solution}

\begin{problem}[4]
    Load the image located at \url{https://math173.github.io/img/dog.jpg} into a computational language of your choice.
    We can think of this grayscale image as a matrix $A\in\mathbb{R}^{m\times n}$. Where each element $A_{ij}$ is
    the intensity of the pixel $A_{ij}$.
\begin{enumerate}[(a)]
    \item Plot\footnote{You can use an SVD function. No need to implement it.} (in decreasing order) the singular values of $A$.
    \item Plot the truncated SVD $A_k$ as an image for $k=5,10,15,50$ and compare it to $A$.
    \item How many entries do we need to store to represent $A_k$ for general $A\in\mathbb{R}^{m\times n}$?
        Using this, interpret $A_k$ as a lossy compression scheme for a matrix $A$.
\end{enumerate}
\end{problem}

\begin{solution}
    \vfill
\end{solution}
\clearpage

\end{document}
