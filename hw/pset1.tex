\documentclass[12pt,letterpaper,cm]{hmcpset}
\usepackage[margin=1in]{geometry}
\usepackage{graphicx}
\usepackage{amsmath,amssymb}
\usepackage{enumerate}

% info for header block in upper right hand corner
\name{\_\_\_\_\_\_}
\class{Math 173}
\assignment{Problem Set 1}
\duedate{September 17, 2018}
\setlength\parindent{0pt}

\begin{document}

\textbf{Note:} You may assume the Axiom of Choice in all these problems, but for this
problem set in particular please point out where you use it. The Extra Credit (EC) problem
is worth as much as a regular problem; even if you don't attempt it, you should read it thoroughly
since these problems will hint at future directions in the course or other interesting topics.

\begin{problem}[1]
    Under the Axiom of Choice, prove that every vector space, infinite dimensional or
    not, has a basis. \textit{Hint:} Use Zorn's Lemma.
\end{problem}

\begin{solution}
    \vfill
\end{solution}
\clearpage

\begin{problem}[2 (Lax 2.1)]
    Given a nonzero $x$ in a vector space $\mathcal{V}$ of arbitrary dimension, show that there
    exists a linear functional $f : \mathcal{V} \to \mathbb{F}$ such that $f(x)\neq 0$.
\end{problem}

\begin{solution}
    \vfill
\end{solution}

\begin{problem}[3]
    Let $C[0,1]$ be the vector space of continuous real functions on the interval $[0,1]$.
    Show that $C[0,1]$ has uncountable Hamel dimension. \textit{Hint:} Consider
    $\{e^{\beta x} : \beta\in[0,1]\}$.
\end{problem}

\begin{solution}
    \vfill
\end{solution}
\clearpage

\begin{problem}[4]
    \begin{enumerate}[(1)]
        \item Prove that the real numbers form an infinite dimensional vector space over the rational numbers.
        \item Let $p\in \mathbb{Q}[x]$ be a rational coefficient polynomial. What is the minimal
            number of evaluations of $p$ needed to uniquely determine $p$ (a) if we are allowed
            to evaluate only at rational points and (b) if we are allowed to evaluate at real points.
    \end{enumerate}
\end{problem}

\begin{solution}
    \vfill
\end{solution}
\clearpage


\begin{problem}[EC]
    Let $x\in[0,1]$ and $X\sim\operatorname{Binomial}(n,x)$ be a binomial
    random variable. Take $f\in C[0,1]$. Observe that
    \[
        \mathbb{E} f(X/n) = \sum_{k=0}^n f\bigl(\tfrac{k}{n}\bigr) {n\choose k} x^k (1-x)^{n-k}.
    \]
    The right hand side of this, as a function of $x$, is called the $n$-th Bernstein
    polynomial of $f$, denoted $B_n(f)$. We will show that $B_n(f) \to f$ uniformly,
    which shows that $\{x^k (1-x)^{n-k} : 0\leq k \leq n\}$ approximates a basis
    for $C[0,1]$ in some sense. This shows that, while $C[0,1]$ has uncountable Hamel dimension,
    it has `almost-countable' dimension, a notion that will be made more precise later in
    the course. (\textbf{Remark:} This also proves of the Wierstrass Approximation Theorem,
    that says that the polynomials are dense in $C[0,1]$.)
    \begin{enumerate}[(1)]
        \item Prove that
            \[
                \lim_{n\to\infty}\sup_{x\in[0,1]}\mathbb{P}(|\tfrac{X}{n} - x|\geq \epsilon) = 0
            \]
            for all $\epsilon > 0$. \textit{Hint:} Recall Chebyshev's
            inequality\footnote{Chebyshev's inequality says that
            $\mathbb{P}(|X - \mathbb{E}X| \geq \gamma)\leq \gamma^{-2}\text{Var}(X)$.} and use a
            diagonalization argument.
        \item Infer that
            \[
                \lim_{n\to\infty}\sup_{x\in[0,1]} \mathbb{P}(|f(\tfrac{X}{n}) - f(x)|\geq \epsilon) = 0\quad\text{to prove}\quad
                \lim_{n\to\infty}\sup_{x\in[0,1]}\mathbb{E}|f(\tfrac{X}{n}) - f(x)| = 0.
            \]
        \item Conclude that $\lim_{n\to\infty}\|B_n(f) - f\|_\infty =
            \lim_{n\to\infty}\sup_{x\in[0,1]}|B_n(f)(x) - f(x)| = 0$.
    \end{enumerate}
\end{problem}

\begin{solution}
    \vfill
\end{solution}
\clearpage

\end{document}
